\documentclass[letterpaper,12pt]{article}
\oddsidemargin -1.0cm \textwidth 16.5cm

\usepackage[T1]{fontenc}
\usepackage[utf8]{inputenc}
\usepackage[spanish]{babel}
\usepackage{amsfonts,setspace}
\usepackage{amsmath}
\usepackage{amssymb, amsmath, amsthm}
\usepackage{comment}
\usepackage{amssymb}
\usepackage{dsfont}
\usepackage{anysize}
\usepackage{multicol}
\usepackage{enumerate}
\usepackage{graphicx}
\usepackage[left=2cm,top=2cm,right=2cm, bottom=2cm]{geometry}
\usepackage{fancyhdr}
\pagestyle{fancy}
\usepackage{wrapfig}
\usepackage{hyperref}



%Teoremas, Lemas, etc.
\theoremstyle{plain}
\newtheorem{teo}{Teorema}
\newtheorem{lem}{Lemma}
\newtheorem{prop}{Proposici\'on}
\newtheorem{cor}{Corolario}
\theoremstyle{definition}
\newtheorem{defi}{Definici\'on}
\newtheorem{eje}{Ejemplo}
\newtheorem{obs}{Observaci\'on}
% fin macros

\decimalpoint

\newcommand{\Z}{\mathbb{Z}}
\newcommand{\Q}{\mathbb{Q}}
\newcommand{\R}{\mathbb{R}}
\newcommand{\RR}{\mathbb{R} \times \mathbb{R}}
\newcommand{\ssi}{\Leftrightarrow}
\newcommand{\si}{\Rightarrow}
\newcommand{\y}{\wedge}
\newcommand{\oo}{\vee}
\newcommand{\Pa}{\mathcal{P}}
\newcommand{\N}{\mathbb{N}}
\newcommand{\C}{\mathbb{C}}

\begin{document}

\fancyhead[L]{\itshape{Facultad de Ciencias F\'isicas y Matem\'aticas}}
\fancyhead[R]{\itshape{Universidad de Chile}}

\begin{minipage}{11.5 cm}
  \begin{flushleft}
    \hspace*{-0.6cm}CC3002 Metodologías de Diseño y Programación 2019-2\\
\end{flushleft}\end{minipage}

\begin{picture}(2,3)
\put(360,-20){\includegraphics[scale=0.18]{./fcfm.pdf}}
\end{picture}

\begin{center}
 \LARGE \bf{Alpaca Emblem}\\
 \normalsize Entrega 1
\end{center}
% Transformada de Laplace
\setlength{\parindent}{0cm}

\subsection*{Datos Personales}
\begin{itemize}
	\item \textbf{Nombre:} Tabita Andrea Catalán Muñoz
	\item \textbf{RUT:} 19.603.186-3
	\item \textbf{Repositorio de GitHub:} \url{https://github.com/tabitaCatalan/AlpacaEmblem}
\end{itemize}

\subsection*{Diagrama UML}
(No logré poner todo junto, tuve problemas para exportar como \texttt{svg} así que lo dejé en \texttt{png})

\begin{figure}[h]
\caption{Diagrama UML Resumido del paquete \texttt{items}}
\centering
\includegraphics[width=\textwidth]{UML/PackageItems.png}
\end{figure}

\begin{figure}[h]
\caption{Diagrama UML Resumido del paquete \texttt{items.magic}}
\centering
\includegraphics[width=0.7\textwidth]{UML/PackageMagic.png}
\end{figure}

\begin{figure}[h]
\caption{Diagrama UML Resumido del paquete \texttt{items.nonMagic}}
\centering
\includegraphics[width=0.7\textwidth]{UML/PackageNonMagic.png}
\end{figure}


\begin{figure}[h]
\caption{Diagrama UML Resumido del paquete \texttt{units}}
\centering
\includegraphics[width=\textwidth]{UML/PackageUnits.png}
\end{figure}


\end{document}